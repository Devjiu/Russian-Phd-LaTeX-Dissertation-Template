\begin{frame}
    \frametitle{Положения, выносимые на защиту}
    \begin{itemize}
        \item Зеркальный спуск для вариационных неравенств с относительно сильно монотонными и относительно ограниченными операторами. Доказана оптимальная оценка на указанном классе с точностью  постоянного множителя.
        \item Адаптивная оценка скорости сходимости зеркального спуска для сильно выпуклых функций с использованием локальных аналогов константы Липшица. 
        \item Введён аналог ослабленного минимума ($\gamma$-роста) с использованием дивергенции Брэгмана. Получена оценка скорости сходимости рестартованного метода зеркального спуска для относительно липшицевых задач оптимизации с относительным $\gamma$-ростом. В ситуации сильно выпуклой прокс-функции предложены адаптивные правила остановки для рестартов.
    \end{itemize}
\end{frame}
\note{
    Проговариваются вслух положения, выносимые на защиту
}

\begin{frame}
    \frametitle{Научная новизна}
    \begin{itemize}
        \item Впервые сформулированы и доказаны оценки для метода зеркального спуска для вариационных неравенств с относительно сильно монотонными и относительно ограниченными операторами.
        \item Впервые сформулированы и доказаны оценки скорости сходимости рестартованного метода зеркального спуска для относительно липшицевых задач оптимизации с относительным $\gamma$-ростом.
        \item Было выполнено оригинальное исследование для уточнения оптимальных оценок зеркального спуска для задач минимизации сильно выпуклых функций с использованием локальных аналогов константы Липшица, что позволило получить обобщение на случай относительно липшицевой постановки.
    \end{itemize}
\end{frame}
\note{
    Проговаривается вслух научная новизна
}

\begin{frame} % публикации на одной странице
% \begin{frame}[t,allowframebreaks] % публикации на нескольких страницах
    \frametitle{Основные публикации}
    % \nocite{vakbib1}%
    % \nocite{vakbib2}%
    % %
    % %% authorwos
    % \nocite{wosbib1}%
    % %
    % %% authorscopus
    % \nocite{scbib1}%
    % %
    % %% authorconf
    % \nocite{confbib1}%
    % \nocite{confbib2}%
    % %
    % %% authorother
    % \nocite{bib1}%
    % \nocite{bib2}%
    \nocite{yakovlev2019algorithms}
    \nocite{Stonyakin_2021}
    \nocite{sharp22}
    \ifnumequal{\value{bibliosel}}{0}{
        \insertbiblioauthor
    }{
        \printbibliography%
    }
\end{frame}
\note{
    Результаты работы опубликованы в N печатных изданиях,
    в~т.\:ч. M реферируемых изданиях.
}

\begin{frame}
    \frametitle{Участие в конференциях}
    \begin{itemize}
        \item проектной смене <<Современные методы теории информации, оптимизации и управления>> в центре <<Сириус>>, 2021
        \item QIPA (Quasilinear Equations, Inverse Problems and Their Applications), 2021
        \item 64-й всероссийской научной конференции МФТИ, 2021
        \item QIPA (Quasilinear Equations, Inverse Problems and Their Applications), 2018
        \item 61-й всероссийской научной конференции МФТИ, 2018
    \end{itemize}
\end{frame}
\note{
    Работа была представлена на ряде конференций.
}

\begin{frame}[plain, noframenumbering] % последний слайд без оформления
    \begin{center}
        \Huge
        Спасибо за внимание!
    \end{center}
\end{frame}
