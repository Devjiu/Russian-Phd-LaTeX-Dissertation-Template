\begin{frame}{2 направление. Случай седловых задач.}
    Будем рассматривать выпукло-вогнутую седловую задачу вида:
    \begin{equation}
    \begin{aligned} 
        f^* = \min_{u \in Q_1} \max_{v \in Q_2} f(u, v),
    \end{aligned}
    \end{equation}
    где $f$ --- относительно сильно выпуклый по $u$ и  относительно сильно вогнутый по $v$.

    Пусть $x = (u, v), y = (z, t)$, также введем оператор 
    $$ 
      g(x) := \Bigg( 
      \begin{aligned}
        f^{'}_{u}(u,v)&&\\
        -f^{'}_{v}(u,v)&&
      \end{aligned}
      \Bigg), 
    $$
    что позволяет перейти к ВН и применить аналогичный метод для получения соответствующей оценки:
    \begin{equation}
    \begin{aligned}
        \max_{v} f(\widehat{u}, v) - \min_{u} f(u, \widehat{v}) \leq \frac{2M^2}{\mu (N+1)}
    \end{aligned}
    \end{equation}
\end{frame}

\begin{frame}
    \frametitle{Ответы на замечания ведущей организации НИИ~<<Рога~и~копыта>>}
    \begin{itemize}
        \item Замечание -- ответ
        \item Замечание -- ответ
        \item Замечание -- ответ
        \item Замечание -- ответ
        \item Замечание -- ответ
    \end{itemize}
\end{frame}
