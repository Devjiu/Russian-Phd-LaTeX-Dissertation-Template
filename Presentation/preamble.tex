\begin{frame}[noframenumbering,plain]
    \setcounter{framenumber}{1}
    \maketitle
\end{frame}

\begin{frame}{Структура работы}
    Работа состоит из 3 глав:
    \begin{itemize}
        \item Первая глава: 
        \begin{itemize}
            \item Обзор методов первого порядка
            \item Распределенная задача минимизации 
            \item Задача минимизации энергии белка
        \end{itemize}
        \item Вторая глава:
        \begin{itemize}
            \item Введение определений и класса вариационных неравенств
            \item Зеркальный спуск для вариационных неравентсв
            \item Адаптивный аналог оптимальной оценки
        \end{itemize}
        \item Третья глава:
        \begin{itemize}
            \item Сравнение сильной выпуклости и острого минимума
            \item Предлагается относительный аналог и обобщение условия острого минимума
            \item Предложен алгоритм рестартов зеркального спуска при дополнительных условиях
        \end{itemize}
    \end{itemize}
\end{frame}
\note{
    Работа состоит из трёх глав.

    \medskip
    В первой главе \dots

    Во второй главе \dots

    Третья глава посвящена \dots
}
