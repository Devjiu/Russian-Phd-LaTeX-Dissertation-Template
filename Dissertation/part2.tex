\chapter{Рестарты зеркального спyска для липшицевых задач с yсловием gamma-роста}\label{ch:ch2}

\section{Просто рестарты}\label{sec:ch3/sect1}

\begin{theorem} \label{vanilla_mirror}
    Пусть $f$ --- является $M$-липщицевой на $Q$ относительно некоторой функции Брегмана $V_d(x, y)$ c 1-сильно выпуклой прокс-функцией $d(x)$. Тогда можно задать метод следующим образом:
    \begin{equation} \label{mirr_upd}
        x_{k+1} = \arg \min_{x \in Q} {f(x_k) + \langle g(x_k), x - x_k \rangle + \frac{1}{h_k} V_d(x, x_k)},
    \end{equation}
    где $\{ h_k \}$ - последовательность размеров шагов.
    Для него справедлива следующая оценка скорости сходимости:
    \begin{equation}
        \min_{0\leq k \leq N} f(x_k) - f(x) \leq \frac{\frac{1}{2} M^2 \sum_{k=0}^N h_k^2 + V(x, x_0)}{\sum_{k=0}^N t_k}
    \end{equation}
\end{theorem}


\begin{remark}
    Если в \ref{vanilla_mirror} выбрать шаг следующим образом:
    \begin{equation} \label{mirr_step}
        h_{k} = \frac{\sqrt{2 V(x_*, x_0)}}{M\sqrt{N}},
    \end{equation}
    то скорость сходимости можно оценить так:
    \begin{equation} \label{mirr_est}
        f(\widehat{x_N}) - f(x_*) \leq \frac{M\sqrt{2V(x_*, x_0)}}{\sqrt{N}}
    \end{equation}
\end{remark}
Если функция обладает дополнительными свойствами, аналогичными <<острому минимуму>>,  то становится возможным применение техники рестартов. Один из возможных аналогов - условие $\gamma$-роста:

\begin{definition} \label{gamma-growth}
   $f$ --- удовлетворяет условию $\gamma$-роста тогда и только тогда:
   \begin{equation}
       f(x) - f(x_*) \geq \mu_{\gamma}(V(x_*,x))^{\gamma/2}
   \end{equation}
\end{definition}

\begin{theorem}
    Пусть $f$ --- удовлетворяет условию $\gamma$-роста (\ref{gamma-growth}) и также является $M$-липщицевой на $Q$ относительно некоторой функции Брегмана $V_d(x, y)$. В таком случае Алгоритм \ref{alg:rest_gamma} достигнет точности $\epsilon$ не более чем за:
    \begin{equation}
       N = \frac{M^2 2^{\gamma}(1 - V(x_*, x_0^0)^{1 - \gamma}) }{\mu_{\gamma}^2 \varepsilon^{(\gamma-1)} (2^{(\gamma-1)} - 1)}
    \end{equation}
    причем будет справедливо неравенство:
    \begin{equation}
       V(x_*, \widehat{x_p}) \leq \varepsilon
    \end{equation}
\end{theorem}

\begin{algorithm}[htp]
   \caption{Рестарты зеркального спуска при условии $\gamma$-роста.}
   \label{alg:rest_gamma}
   \begin{algorithmic}[1]
       \REQUIRE $\varepsilon > 0$
       \STATE $p=0,V(x_*, x_0)=V(x_*,x_0^0)$.
       \REPEAT
       \STATE $x_{p}$ --- результат работы метода \ref{mirr_upd} с шагом \ref{mirr_step} и количеством шагов $N_{p} = \frac{M^2 2^{\gamma}}{\mu_{\gamma}^2 2^{p(1 - \gamma)}} V(x_*, x_0^0)^{1 - \gamma}$
       \STATE $x_0 = \widehat{x_p}$.
       \STATE $V(x_*, x_0) \leftarrow \frac{1}{2^{p+1}}V_{0}(x_*, x_0^0)$.
       \STATE $p=p+1$.
       \UNTIL			
       $p > \log_2\left(\frac{V(x_*, x_0^0)}{\varepsilon}\right).$	
       \ENSURE $x_p$.
   \end{algorithmic}
\end{algorithm}

\begin{proof}
   Объединим в систему свойство \ref{gamma-growth} и оценку \ref{mirr_est}. В доказательстве используется обозначение $x_m^n$, где $m - 0...N$ соответсвует номеру итерации и $n - 0...p$ соответсвует номеру рестарта: 
   $$
       \mu_{\gamma}(V(x_*, \widehat{x_N^0}))^{\frac{\gamma}{2}} \leq f(\widehat{x_N^0}) - f(x_*) \leq \frac{M\sqrt{V(x_*, x_0^0)}}{\sqrt{N}}
   $$
   $$
       \mu_{\gamma}(V(x_*, \widehat{x_N^0}))^{\frac{\gamma}{2}} \leq \frac{M\sqrt{V(x_*, x_0^0)}}{\sqrt{N}}
   $$
   $$
       (V(x_*, \widehat{x_N^0}))^{\frac{\gamma}{2}} \leq \frac{M\sqrt{V(x_*, x_0^0)}}{\mu_{\gamma}\sqrt{N}}
   $$
   $$
       V(x_*, \widehat{x_N^0}) \leq (\frac{M}{\mu_{\gamma}\sqrt{ N}})^{\frac{2}{\gamma}} (V(x_*, x_0^0))^{\frac{1}{\gamma}}
   $$
   $$
       V(x_*, \widehat{x_N^0}) \leq V(x_*, x_0^0) (\frac{M}{\mu_{\gamma}\sqrt{N}})^{\frac{2}{\gamma}} (V(x_*, x_0^0))^{\frac{1}{\gamma} - 1}
   $$
   Теперь можно оценить необходимое количество итераций для 0 запуска:
   $$
       (\frac{M}{\mu_{\gamma}\sqrt{N}})^{\frac{2}{\gamma}} (V(x_*, x_0^0))^{\frac{1}{\gamma} - 1} \leq \frac{1}{2} 
   $$
   $$
       (\frac{M}{\mu_{\gamma}})^{\frac{2}{\gamma}} (V(x_*, x_0^0))^{\frac{1}{\gamma} - 1} \leq \frac{1}{2} N^{\frac{1}{\gamma}} 
   $$
   $$
       (\frac{M}{\mu_{\gamma}})^{\frac{2}{\gamma}} (V(x_*, x_0^0))^{\frac{1}{\gamma} - 1} \leq \frac{1}{2} N^{\frac{1}{\gamma}} 
   $$
   $$
       \frac{1}{2} N^{\frac{1}{\gamma}} \geq (\frac{M}{\mu_{\gamma}})^{\frac{2}{\gamma}} (V(x_*, x_0^0))^{\frac{1}{\gamma} - 1}  
   $$
   $$
       N \geq 2 ^ {\gamma} \frac{M^2}{\mu_{\gamma}^2} (V(x_*, x_0^0))^{1 - \gamma}  
   $$
   $$
       N \geq \frac{M^2 2^{\gamma}}{\mu_{\gamma}^2} (V(x_*, x_0^0))^{1 - \gamma}  
   $$
   Проверим поведение для нескольких последующих рестартов. Для 1-го запуска $x_0^1 = \widehat{x_N^0}$:
   $$
       V(x_*, \widehat{x_N^1}) \leq \frac{1}{2} V(x_*, x_0^1) = \frac{1}{2} V(x_*, \widehat{x_N^0}) \leq (\frac{1}{2})^2 V(x_*, x_0^0) 
   $$
   после:
   $$
       N_1 \geq \frac{M^2 2^{\gamma}}{\mu_{\gamma}^2} (V(x_*, x_0^1))^{1 - \gamma} \geq \frac{M^2 2^{\gamma}}{\mu_{\gamma}^2} (V(x_*, \widehat{x_N^0}))^{1 - \gamma} 
   $$
   при $\gamma > 1$:
   $$
       V(x_*, \widehat{x_N^0}) \leq \frac{1}{2}V(x_*, x_0^0)  
   $$
   $$
       (V(x_*, \widehat{x_N^0}))^{\gamma - 1} \leq (\frac{1}{2} V(x_*, x_0^0))^{\gamma - 1}
   $$
   $$
        (\frac{1}{2} V(x_*, x_0^0))^{1 - \gamma} \leq (V(x_*, \widehat{x_N^0}))^{1 - \gamma}
   $$
   соответсвенно:
   $$
       N_1 \geq \frac{M^2 2^{\gamma}}{\mu_{\gamma}^2} (V(x_*, \widehat{x_N^0}))^{1 - \gamma} \geq \frac{M^2 2^{\gamma}}{\mu_{\gamma}^2} (\frac{1}{2} V(x_*, x_0^0))^{1 - \gamma} = \frac{M^2 2^{\gamma}}{\mu_{\gamma}^2 2^{1-\gamma}} (V(x_*, x_0^0))^{1 - \gamma}
   $$
   Для 2-го запуска $x_0^2 = \widehat{x_N^1}$:
   $$
       V(x_*, \widehat{x_N^2}) \leq \frac{1}{2} V(x_*, x_0^2) \leq (\frac{1}{2})^3 V(x_*, x_0^0) 
   $$
   после:
   $$
       N_2 \geq \frac{M^2 2^{\gamma}}{\mu_{\gamma}^2} (V(x_*, x_0^2))^{1 - \gamma} = \frac{M^2 2^{\gamma}}{\mu_{\gamma}^2} (V(x_*, \widehat{x_N^1}))^{1 - \gamma} \geq \frac{M^2 2^{\gamma}}{\mu_{\gamma}^2 2^{1 - \gamma}} (V(x_*, x_0^1))^{1 - \gamma} \geq \frac{M^2 2^{\gamma}}{\mu_{\gamma}^2 2^{2(1 - \gamma)}} (V(x_*, x_0^0))^{1 - \gamma} 
   $$
   Для $(p-1)$-го запуска:
   \begin{equation} \label{v_seq}
       V(x_*, \widehat{x_N^{p-1}}) \leq \frac{1}{2^p} V(x_*, x_0^0)
   \end{equation}
   после:
   \begin{equation} \label{n_seq}
       N_{p-1} \geq \frac{M^2 2^{\gamma}}{\mu_{\gamma}^2 2^{(p - 1)(1 - \gamma)}} (V(x_*, x_0^0))^{1 - \gamma}
   \end{equation}
   Используя найденную зависимость \ref{v_seq}, проведем оценку общего числа обращений к оракулу:
   $$
       \sum_{k=1}^{p - 1} N_k \geq \frac{M^2 2^{\gamma}}{\mu_{\gamma}^2} (V(x_*, x_0^0))^{1 - \gamma} (1 + 2^{(\gamma-1)} + 2^{2(\gamma - 1)} + ... + 2^{(p-1)(\gamma - 1)}) = \frac{M^2 2^{\gamma}}{\mu_{\gamma}^2} \frac{1 - 2^{(p-1)(\gamma-1)}}{1 - 2^{(\gamma-1)}} (V(x_*, x_0^0))^{1 - \gamma}
   $$
   Если задать ограничение для $V(x_*, \widehat{x_N^{p-1}})$:
   $$
       V(x_*, \widehat{x_N^{p-1}}) \leq \frac{1}{2^p} V(x_*, x_0^0) \leq \varepsilon
   $$
   $$
        2^p \geq \frac{1}{\varepsilon} V(x_*, x_0^0)
   $$
   Откуда получаем оценку для количества рестартов:
   \begin{equation}
        p \geq \log_2{\frac{V(x_*, x_0^0)}{\varepsilon}}
   \end{equation}
   Используем это для оценки общего количества обращений к <<оракулу>>:
   $$
       \sum_{k=1}^{p} N_k \geq \frac{M^2 2^{\gamma}}{\mu_{\gamma}^2} \frac{1 - 2^{p(\gamma-1)}}{1 - 2^{(\gamma-1)}} (V(x_*, x_0^0))^{1 - \gamma} \geq \frac{M^2 2^{\gamma}}{\mu_{\gamma}^2 (1 - 2^{(\gamma-1)})} (1 - \frac{V(x_*, x_0^0)^{(\gamma-1)}}{\varepsilon^{(\gamma-1)}}) (V(x_*, x_0^0))^{1 - \gamma} =
   $$
   $$
       = \frac{M^2 2^{\gamma}}{\mu_{\gamma}^2 (2^{(\gamma-1)} - 1)} (\frac{V(x_*, x_0^0)^{(\gamma-1)}}{\varepsilon^{(\gamma-1)}} - 1) (V(x_*, x_0^0))^{1 - \gamma} = \frac{M^2 2^{\gamma}(1 - V(x_*, x_0^0)^{1 - \gamma}) }{\mu_{\gamma}^2 \varepsilon^{(\gamma-1)} (2^{(\gamma-1)} - 1)} 
   $$
   Таким образом получаем оценку:
   $$
       \mathcal{O} (\frac{M^2 2^{\gamma}(1 - V(x_*, x_0^0)^{1 - \gamma}) }{\mu_{\gamma}^2 \varepsilon^{(\gamma-1)} (2^{(\gamma-1)} - 1)})
   $$
\end{proof}

\FloatBarrier