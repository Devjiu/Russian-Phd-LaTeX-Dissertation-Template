
{\actuality} 
В огромном количестве прикладных задач возникает необходимость в подходящих алгоритмах непрерывной оптимизации. Очень важна эффективность применения тех или иных методов нахождения оптимального решения. В широком смысле под эффективностью понимается время, необходимое для получения достаточно хорошего решения. Однако, подобный параметр зависит от огромного количества аспектов, не имеющих прямого отношения к алгоритму поиска достаточно хорошего решения. Такими аспектами могут являться мощность вычислительного устройства, доступная точность представления чисел, необходимое количество памяти и т.д. Чтобы разделить технический и теоретический аспекты под эффективностью метода на определенном классе задач (набор задач объединенных некоторой характеристикой) часто понимают именно эффективность (оптимальность) в смысле Бахвалова-Немировского \cite{Nemirovski1979} --- число обращений по ходу работы метода к \textit{оракулу}. Оракулом называется подпрограмма расчета значений целевой функции (градиента или иных метрик более высокого порядка) с необходимой точностью.

Таким образом аналитическая сложность оптимизационной задачи для метода на заданном классе характеризуется необходимым количеством запросов к оракулу для нахождения приближенного решения с заранее заданной точностью $\varepsilon > 0$.

Такой оракульный подход также называют \textit{концепцией чёрного ящика}, что очень удобно для анализа сложности и сравнения методов. Тут стоит обратить внимание, что ответы оракула являются локальными и не накладывают никаких ограничений, не требуют каких-либо свойств от анализируемого функционала. Также отметим, что существует несколько типов оценок:
\begin{itemize}
    \item верхняя --- количество обращений к оракулу не более данного значения, (хуже не будет)
    \item нижняя --- не менее данного числа обращений, (лучше не будет)
    \item оптимальная --- описывает случай, когда нижняя и верхняя оценки совпадают с точностью до константных множителей.
\end{itemize}

Стоит упомянуть, что ранее были более востребованы небольшой востребованности. Задачей \textit{небольшой размерности} обычно называют задачу, когда возможно $N \geq n$, где $N$ --- количество обращений к оракулу, а $n$ --- размерность пространства. Для данных задач популярны методы внутренней точки, к которым относятся метод центров тяжести и метод эллипсоидов. Однако их оптимальные оценки $N \sim \mathcal{O}\left(n^2\right)$ имеют существенную зависимость от размерности пространства. \cite{bubeck_2015}

Эту зависимость вместе с быстрым ростом размерности задач называют основной причиной переноса фокуса большинства исследований с методов внутренней точки к градиентным методам. Также стоит отметить, что методы субградиентного (как и градиентного) типа обладают небольшими затратами памяти на итерациях. Лучшие верхние оценки в гладком случае получены для ускоренных методов. \cite{Nesterov1983} 

\begin{table}[h]
    \caption{Оптимальные оценки количества обращений к субградиенту.}
    \label{est_tbl}
    \centering
    \begin{tabular}{|c|c|c|}
        \hline
         & \makecell{$|f(y) - f(x)| \leq$ \\ $\leq M \| y - x \|$} & \makecell{$\|\nabla f(y) - \nabla f(x)\|_* \leq $\\ $\leq L \| y - x \|$} \\
        \hline
        $f(x)$ -- выпукла & $\mathcal{O} \left( \frac{M^2 R^2}{\varepsilon^2} \right)$ & $\mathcal{O} \left( \sqrt{\frac{L R^2}{\varepsilon}} \right)$ \\
        \hline
        \makecell{$f(x)$ -- $\mu$-сильно \\ выпукла в $\| \cdot \|$ - норме} & $\mathcal{O} \left( \frac{M^2}{\mu \varepsilon} \right)$ & $\mathcal{O} \left( \sqrt{\frac{L}{\mu}} \left[\ln{\frac{\mu R^2}{\varepsilon}}\right] \right)$ \\
        \hline
    \end{tabular}
\end{table}
В таблице \ref{est_tbl} $R$ --- это <<расстояние>> между начальной точкой и ближайшим решением. 

Во многих областях, таких как машинное обучение, анализ данных, оценка риска и некоторые физические приложения возникают функционалы не обладающие достаточными свойствами для эффективной их минимизации методами гладкой оптимизации. По данной причине достаточно актуальны вопросы разработки эффективных методов для работы в негладкой постановке для которых верхние оценки не настолько оптимистичны. 

Для улучшения оценок для негладких задач существует несколько подходов, например выделение специальных подклассов, таких как условие острого минимума, предложенное в конце 1960-х годов Б.Т. Поляком. Стоит подчеркнуть, что подобные условия зачастую выставляют более жесткие требования к задаче, например, острый минимум подразумевает доступность минимального значения функции. 

Популярность набирают различные обобщения условий в оптимальных методах, обладающих приемлемой скоростью сходимости. Возникает естественное желание воспользоваться рядом значимых результатов, таких как оптимальная оценка скорости сходимости на классе липшицевых и сильно выпуклых минимизационных задач, которая достигается именно для субградиентного метода \cite{Bach_2012}. 

Отметим важное и развиваемое в данной работе обобщение на случай задачи с аналогом условия Липшица относительно некоторой выпуклой прокс-функции (относительная липшицевость), которая, в отличие от классической постановки, не обязана удовлетворять условию сильной выпуклости относительно нормы \cite{AdaMirr_2021,Lu_2018,Zhou_NIPS_2020}. В данной работе исследуется оценка скорости сходимости субградиентного метода для сильно выпуклых задач с аналогичным предположением об относительной липшицевости. Точнее рассматривается вариант субградиентного метода на классе относительно ограниченных и относительно сильно монотонных вариационных неравенств, а также класс относительно сильно выпукло-вогнутых седловых задач с соответствующими условиями относительной липшицевости функционалов. В свою очередь вариационные неравенства являются важной вехой для работы с, например, лагранжевыми седловыми задачами. 

Далее, немалую популярность в работах по оптимизации получило упомянутое выше недавно предложенное понятие относительной гладкости функций (см. работы \cite{Bauschke,Drag,Dragomir,Lu_Nesterov_2018}, а также приведённые в них ссылки), которое позволило существенно расширить класс задач выпуклой оптимизации по сравнению со стандартным предположением о липшицевости градиента с гарантией оценки скорости сходимости $\mathcal{O}(N^{-1})$ (здесь и далее $N$ --- количество итераций), которая может считаться оптимальной для такого широкого класса задач \cite{Dragomir}. 

Таким образом, данная работа обогащает ряд важных практических задач набором методов с оптимальными на их классе оценками.

% {\progress}
% Этот раздел должен быть отдельным структурным элементом по
% ГОСТ, но он, как правило, включается в описание актуальности
% темы. Нужен он отдельным структурынм элемементом или нет ---
% смотрите другие диссертации вашего совета, скорее всего не нужен.

{\aim} данной работы является исследование и совершенствование методов оптимизации для задач, не обладающих достаточной гладкостью для применения классических методов. Подобные задачи возникают в различных областях, таких как белковый фолдинг и машинное обучение. Существенно, что исследуемые задачи имеют большую размерность. В данной работе автор для метода зеркального спуска расширяет доступный класс задач и рассматривает аналоги условия Липшица, которые позволяют сохранить свойственную липшицевым задачам оптимальные оценки скорости сходимости. Сделан существенный акцент на экспериментальной составляющей.

{\novelty}
\begin{enumerate}[beginpenalty=10000] % https://tex.stackexchange.com/a/476052/104425
  \item Впервые сформулированы и доказаны оценки для метода зеркального спуска для вариационных неравенств с относительно сильно монотонными и относительно ограниченными операторами.
  \item Впервые сформулированы и доказаны оценки скорости сходимости рестартованного метода зеркального спуска для относительно липшицевых задач оптимизации с относительным $\gamma$-ростом.
  \item Было выполнено оригинальное исследование для уточнения оптимальных оценок зеркального спуска для задач минимизации сильно выпуклых функций с использованием локальных аналогов константы Липшица, что позволило получить обобщение на случай относительно липшицевой постановки. 
\end{enumerate}

{\defpositions}
\begin{enumerate}[beginpenalty=10000] % https://tex.stackexchange.com/a/476052/104425
  \item Предложен вариант метода зеркального спуска для вариационных неравенств с относительно сильно монотонными и относительно ограниченными операторами. Доказана инвариантная по размерности пространства оценка скорости сходимости этого метода, оптимальная на указанном классе задач с точностью до умножения на постоянный множитель.
  \item Получена адаптивная оценка скорости сходимости зеркального спуска для задач минимизации сильно выпуклых функций с использованием локальных аналогов константы Липшица. При этом сохраняется оптимальность этой оценки на классе сильно выпуклых липшицевых задач с точностью до умножения на константу. В частности, это позволяет работать и с задачами, не удовлетворяющими условию относительной липшицевости.
  \item Введён аналог ослабленного минимума ($\gamma$-роста) с использованием дивергенции Брэгмана. Получена оценка скорости сходимости рестартованного метода зеркального спуска для относительно липшицевых задач оптимизации с относительным $\gamma$-ростом. В ситуации сильно выпуклой прокс-функции предложены адаптивные правила остановки для рестартов исследуемого метода зеркального спуска и получен результат о его скорости сходимости.
\end{enumerate}

{\probation}
Основные результаты работы докладывались~на:
\begin{itemize}
    \item проектной смене <<Современные методы теории информации, оптимизации и управления>> в центре <<Сириус>>, 2021
    \item QIPA (Quasilinear Equations, Inverse Problems and Their Applications), 2021
    \item 64-й всероссийской научной конференции МФТИ, 2021
    \item QIPA (Quasilinear Equations, Inverse Problems and Their Applications), 2018
    \item 61-й всероссийской научной конференции МФТИ, 2018
\end{itemize}

{\contribution} Ключевые результаты получены и доказаны автором лично. Также разработана библиотека для анализа и проверки методов оптимизации, обеспечивающая необходимую гибкость настройки. 

\ifnumequal{\value{bibliosel}}{0}
{%%% Встроенная реализация с загрузкой файла через движок bibtex8. (При желании, внутри можно использовать обычные ссылки, наподобие `\cite{vakbib1,vakbib2}`).
    {\publications} Основные результаты по теме диссертации изложены
    в~XX~печатных изданиях,
    X из которых изданы в журналах, рекомендованных ВАК,
    X "--- в тезисах докладов.
}%
{%%% Реализация пакетом biblatex через движок biber
    \begin{refsection}[bl-author, bl-registered]
        % Это refsection=1.
        % Процитированные здесь работы:
        %  * подсчитываются, для автоматического составления фразы "Основные результаты ..."
        %  * попадают в авторскую библиографию, при usefootcite==0 и стиле `\insertbiblioauthor` или `\insertbiblioauthorgrouped`
        %  * нумеруются там в зависимости от порядка команд `\printbibliography` в этом разделе.
        %  * при использовании `\insertbiblioauthorgrouped`, порядок команд `\printbibliography` в нём должен быть тем же (см. biblio/biblatex.tex)
        %
        % Невидимый библиографический список для подсчёта количества публикаций:
        \printbibliography[heading=nobibheading, section=1, env=countauthorvak,          keyword=biblioauthorvak]%
        \printbibliography[heading=nobibheading, section=1, env=countauthorwos,          keyword=biblioauthorwos]%
        \printbibliography[heading=nobibheading, section=1, env=countauthorscopus,       keyword=biblioauthorscopus]%
        \printbibliography[heading=nobibheading, section=1, env=countauthorconf,         keyword=biblioauthorconf]%
        \printbibliography[heading=nobibheading, section=1, env=countauthorother,        keyword=biblioauthorother]%
        \printbibliography[heading=nobibheading, section=1, env=countregistered,         keyword=biblioregistered]%
        \printbibliography[heading=nobibheading, section=1, env=countauthorpatent,       keyword=biblioauthorpatent]%
        \printbibliography[heading=nobibheading, section=1, env=countauthorprogram,      keyword=biblioauthorprogram]%
        \printbibliography[heading=nobibheading, section=1, env=countauthor,             keyword=biblioauthor]%
        \printbibliography[heading=nobibheading, section=1, env=countauthorvakscopuswos, filter=vakscopuswos]%
        \printbibliography[heading=nobibheading, section=1, env=countauthorscopuswos,    filter=scopuswos]%
        %
        \nocite{*}%
        %
        {\publications} Основные результаты по теме диссертации изложены в~\arabic{citeauthor}~печатных изданиях,
        \arabic{citeauthorvak} из которых изданы в журналах, рекомендованных ВАК\sloppy%
        \ifnum \value{citeauthorscopuswos}>0%
            , \arabic{citeauthorscopuswos} "--- в~периодических научных журналах, индексируемых Web of~Science или Scopus\sloppy%
        \fi%
        \ifnum \value{citeauthorconf}>0%
            , \arabic{citeauthorconf} "--- в~тезисах докладов.
        \else%
            .
        \fi%
        \ifnum \value{citeregistered}=1%
            \ifnum \value{citeauthorpatent}=1%
                Зарегистрирован \arabic{citeauthorpatent} патент.
            \fi%
            \ifnum \value{citeauthorprogram}=1%
                Зарегистрирована \arabic{citeauthorprogram} программа для ЭВМ.
            \fi%
        \fi%
        \ifnum \value{citeregistered}>1%
            Зарегистрированы\ %
            \ifnum \value{citeauthorpatent}>0%
            \formbytotal{citeauthorpatent}{патент}{}{а}{}\sloppy%
            \ifnum \value{citeauthorprogram}=0 . \else \ и~\fi%
            \fi%
            \ifnum \value{citeauthorprogram}>0%
            \formbytotal{citeauthorprogram}{программ}{а}{ы}{} для ЭВМ.
            \fi%
        \fi%
        % К публикациям, в которых излагаются основные научные результаты диссертации на соискание учёной
        % степени, в рецензируемых изданиях приравниваются патенты на изобретения, патенты (свидетельства) на
        % полезную модель, патенты на промышленный образец, патенты на селекционные достижения, свидетельства
        % на программу для электронных вычислительных машин, базу данных, топологию интегральных микросхем,
        % зарегистрированные в установленном порядке.(в ред. Постановления Правительства РФ от 21.04.2016 N 335)
    \end{refsection}%
    \begin{refsection}[bl-author, bl-registered]
        % Это refsection=2.
        % Процитированные здесь работы:
        %  * попадают в авторскую библиографию, при usefootcite==0 и стиле `\insertbiblioauthorimportant`.
        %  * ни на что не влияют в противном случае
        % \nocite{vakbib2}%vak
        % \nocite{patbib1}%patent
        % \nocite{progbib1}%program
        % \nocite{bib1}%other
        % \nocite{confbib1}%conf
    \end{refsection}%
        %
        % Всё, что вне этих двух refsection, это refsection=0,
        %  * для диссертации - это нормальные ссылки, попадающие в обычную библиографию
        %  * для автореферата:
        %     * при usefootcite==0, ссылка корректно сработает только для источника из `external.bib`. Для своих работ --- напечатает "[0]" (и даже Warning не вылезет).
        %     * при usefootcite==1, ссылка сработает нормально. В авторской библиографии будут только процитированные в refsection=0 работы.
}

