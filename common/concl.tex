%% Согласно ГОСТ Р 7.0.11-2011:
%% 5.3.3 В заключении диссертации излагают итоги выполненного исследования, рекомендации, перспективы дальнейшей разработки темы.
%% 9.2.3 В заключении автореферата диссертации излагают итоги данного исследования, рекомендации и перспективы дальнейшей разработки темы.
\begin{enumerate}
  \item Доказаны оптимальные оценки для зеркального спуска для вариационных неравенств с относительно сильно монотонными и относительно ограниченными операторами с точностью до умножения на постоянный множитель.
  \item Получена адаптивная оценка скорости сходимости зеркального спуска для задач минимизации сильно выпуклых функций с использованием локальных аналогов константы Липшица.  
  \item Доказана оценка скорости сходимости рестартованного метода зеркального спуска для относительно липшицевых задач оптимизации с относительным  $\gamma$-ростом.
  \item Предложены адаптивные правила остановки для рестартов исследуемого метода зеркального спуска в предположении липшицевости и 
  $\gamma$-ростом целевой функции и получены теоретические оценки сложности такой  процедуры. 
  \item Проведено исследование и сравнение практической работы методов безградиентного, градиентного и квазинютоновского типов для минимизации функционала OPLS force field, отвечающего за минимизацию энергии белка.
  \item Реализована вспомогательная библиотека для экспериментальной проверки исследуемых в работе субградиентных методов. 
\end{enumerate}
